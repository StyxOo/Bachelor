
%%% File-Information {{{
%%% Filename: template_bericht.tex
%%% Purpose: lab report, technical report, project report
%%% Time-stamp: <2004-06-30 18:19:32 mp>
%%% Authors: The LaTeX@TUG-Team [http://latex.tugraz.at/]:
%%%          Karl Voit (vk), Michael Prokop (mp), Stefan Sollerer (ss)
%%% History:
%%%   20050914 (ss) correction of "backref=true" to "backref" due to hyperref documentation
%%%   20040630 (mp) added comments to foldmethod at end of file
%%%   20040625 (vk,ss) initial version
%%%
%%% Notes:
%%%
%%%
%%%
%%% }}}
%%%%%%%%%%%%%%%%%%%%%%%%%%%%%%%%%%%%%%%%%%%%%%%%%%%%%%%%%%%%%%%%%%%%%%%%%%%%%%%%
%%% main document {{{

\documentclass[
a4paper,     %% defines the paper size: a4paper (default), a5paper, letterpaper, ...
% landscape,   %% sets the orientation to landscape
% twoside,     %% changes to a two-page-layout (alternatively: oneside)
% twocolumn,   %% changes to a two-column-layout
% headsepline, %% add a horizontal line below the column title
% footsepline, %% add a horizontal line above the page footer
% titlepage,   %% only the titlepage (using titlepage-environment) appears on the first page (alternatively: notitlepage)
% parskip,     %% insert an empty line between two paragraphs (alternatively: halfparskip, ...)
% leqno,       %% equation numbers left (instead of right)
% fleqn,       %% equation left-justified (instead of centered)
% tablecaptionabove, %% captions of tables are above the tables (alternatively: tablecaptionbelow)
% draft,       %% produce only a draft version (mark lines that need manual edition and don't show graphics)
% 10pt         %% set default font size to 10 point
% 11pt         %% set default font size to 11 point
12pt         %% set default font size to 12 point
]{scrartcl}  %% article, see KOMA documentation (scrguide.dvi)



%%%%%%%%%%%%%%%%%%%%%%%%%%%%%%%%%%%%%%%%%%%%%%%%%%%%%%%%%%%%%%%%%%%%%%%%%%%%%%%%
%%%
%%% packages
%%%

%%%
%%% encoding and language set
%%%

%%% ngerman: language set to new-german
%\usepackage{ngerman}

%%% babel: language set (can cause some conflicts with package ngerman)
%%%        use it only for multi-language documents or non-german ones
\usepackage[english]{babel}

%%% inputenc: coding of german special characters
\usepackage[utf8]{inputenc}

%%% fontenc, ae, aecompl: coding of characters in PDF documents
\usepackage[T1]{fontenc}
\usepackage{ae,aecompl}

%%%
%%% technical packages
%%%

%%% amsmath, amssymb, amstext: support for mathematics
%\usepackage{amsmath,amssymb,amstext}

%%% psfrag: replace PostScript fonts
\usepackage{psfrag}

%%% listings: include programming code
%\usepackage{listings}

%%% units: technical units
%\usepackage{units}

%%%
%%% layout
%%%

%%% scrpage2: KOMA heading and footer
%%% Note: if you don't use this package, please remove 
%%%       \pagestyle{scrheadings} and corresponding settings
%%%       below too.
%%% \usepackage[automark]{scrpage2}


%%%
%%% PDF
%%%

\usepackage{ifpdf}

%%% Should be LAST usepackage-call!
%%% For docu on that, see reference on package ``hyperref''
\ifpdf%   (definitions for using pdflatex instead of latex)

  %%% graphicx: support for graphics
  \usepackage[pdftex]{graphicx}

  \pdfcompresslevel=9

  %%% hyperref (hyperlinks in PDF): for more options or more detailed
  %%%          explanations, see the documentation of the hyperref-package
  \usepackage[%
    %%% general options
    pdftex=true,      %% sets up hyperref for use with the pdftex program
    %plainpages=false, %% set it to false, if pdflatex complains: ``destination with same identifier already exists''
    %
    %%% extension options
    backref,      %% adds a backlink text to the end of each item in the bibliography
    pagebackref=false, %% if true, creates backward references as a list of page numbers in the bibliography
    colorlinks=false,   %% turn on colored links (true is better for on-screen reading, false is better for printout versions)
    %
    %%% PDF-specific display options
    bookmarks=true,          %% if true, generate PDF bookmarks (requires two passes of pdflatex)
    bookmarksopen=false,     %% if true, show all PDF bookmarks expanded
    bookmarksnumbered=false, %% if true, add the section numbers to the bookmarks
    %pdfstartpage={1},        %% determines, on which page the PDF file is opened
    pdfpagemode=None         %% None, UseOutlines (=show bookmarks), UseThumbs (show thumbnails), FullScreen
  ]{hyperref}


  %%% provide all graphics (also) in this format, so you don't have
  %%% to add the file extensions to the \includegraphics-command
  %%% and/or you don't have to distinguish between generating
  %%% dvi/ps (through latex) and pdf (through pdflatex)
  \DeclareGraphicsExtensions{.pdf}

\else %else   (definitions for using latex instead of pdflatex)

  \usepackage[dvips]{graphicx}

  \DeclareGraphicsExtensions{.eps}

  \usepackage[%
    dvips,           %% sets up hyperref for use with the dvips driver
    colorlinks=false %% better for printout version; almost every hyperref-extension is eliminated by using dvips
  ]{hyperref}

\fi


%%% sets the PDF-Information options
%%% (see fields in Acrobat Reader: ``File -> Document properties -> Summary'')
%%% Note: this method is better than as options of the hyperref-package (options are expanded correctly)
\hypersetup{
  pdftitle={}, %%
  pdfauthor={}, %%
  pdfsubject={}, %%
  pdfcreator={Accomplished with LaTeX2e and pdfLaTeX with hyperref-package.}, %% 
  pdfproducer={}, %%
  pdfkeywords={} %%
}


%%% rename table of contants
%%% replace    english  with the language you use
\addto\captionsenglish{
  \renewcommand{\contentsname}
    {Table of contents} % Title goes in here
}


%%%%%%%%%%%%%%%%%%%%%%%%%%%%%%%%%%%%%%%%%%%%%%%%%%%%%%%%%%%%%%%%%%%%%%%%%%%%%%%%
%%%
%%% user defined commands
%%%

%%% \mygraphics{}{}{}
%% usage:   \mygraphics{width}{filename_without_extension}{caption}
%% example: \mygraphics{0.7\textwidth}{rolling_grandma}{This is my grandmother on inlinescates}
%% requires: package graphicx
%% provides: including centered pictures/graphics with a boldfaced caption below
%% 
\newcommand{\mygraphics}[3]{
  \begin{center}
    \includegraphics[width=#1, keepaspectratio=true]{#2} \\
    \textbf{#3}
  \end{center}
}

%%%%%%%%%%%%%%%%%%%%%%%%%%%%%%%%%%%%%%%%%%%%%%%%%%%%%%%%%%%%%%%%%%%%%%%%%%%%%%%%
%%%
%%% define the titlepage
%%%

% \subject{}   %% subject which appears above titlehead
% \titlehead{} %% special heading for the titlepage

%%% title
\title{Showcase of d3.js and its possibilities in infographics using refugee data of the current Ukraine conflict}

%%% author(s)
\author{Luis Rothenh\"ausler (20202459)}

%%% date
\date{Brandenburg, \today{}}

% \publishers{}

% \thanks{} %% use it instead of footnotes (only on titlepage)

% \dedication{} %% generates a dedication-page after titlepage


%%% uncomment following lines, if you want to:
%%% reuse the maketitle-entries for hyperref-setup
%\newcommand\org@maketitle{}
%\let\org@maketitle\maketitle
%\def\maketitle{%
%  \hypersetup{
%    pdftitle={\@title},
%    pdfauthor={\@author}
%    pdfsubject={\@subject}
%  }%
%  \org@maketitle
%}


%%%%%%%%%%%%%%%%%%%%%%%%%%%%%%%%%%%%%%%%%%%%%%%%%%%%%%%%%%%%%%%%%%%%%%%%%%%%%%%%
%%%
%%% set heading and footer
%%%

%%% scrheadings default: 
%%%      footer - middle: page number
%%% \pagestyle{scrheadings}

%%% user specific
%%% usage:
%%% \position[heading/footer for the titlepage]{heading/footer for the rest of the document}

%%% heading - left
% \ihead[]{}

%%% heading - center
% \chead[]{}

%%% heading - right
% \ohead[]{}

%%% footer - left
% \ifoot[]{}

%%% footer - center
% \cfoot[]{}

%%% footer - right
% \ofoot[]{}



%%%%%%%%%%%%%%%%%%%%%%%%%%%%%%%%%%%%%%%%%%%%%%%%%%%%%%%%%%%%%%%%%%%%%%%%%%%%%%%%
%%%
%%% begin document
%%%

\begin{document}

% \pagenumbering{roman} %% small roman page numbers

%%% include the title
% \thispagestyle{empty}  %% no header/footer (only) on this page
 \maketitle

%%% start a new page and display the table of contents
% \newpage
\tableofcontents

%%% start a new page and display the list of figures
% \newpage
% \listoffigures

%%% start a new page and display the list of tables
% \newpage
% \listoftables

%%% display the main document on a new page 
% \newpage

% \pagenumbering{arabic} %% normal page numbers (include it, if roman was used above)

%%%%%%%%%%%%%%%%%%%%%%%%%%%%%%%%%%%%%%%%%%%%%%%%%%%%%%%%%%%%%%%%%%%%%%%%%%%%%%%%
%%%
%%% begin main document
%%% structure: \section \subsection \subsubsection \paragraph \subparagraph
%%%
\newpage

\section{Introduction}
The postmodern world produces huge amounts of data every second. Analyzing this data leads to better informed decision making in every sector. Yet the wast amounts of created data are often hard to comprehend with the human mind. Infographics is about finding ways to represent this data in visually appealing, yet easily understandable visual representations. Doing this quickly and always up to date can be crucial. There are many tools available to help with the creating of infographics. This thesis will be a deep dive into the possibilities of one of these tools, the 'd3.js'(D3) library for JavaScript. To show its capabilities we will create graphics, giving an overview of the Ukraine as a sovereign nation and the current events unfolding, using publicly available data.

\section{Basics}
This chapter gives an overview of the different aspects and concepts necessary to understand and follow the thesis.

\paragraph{Infographics}
"Infographics (a clipped compound of "information" and "graphics") are graphic visual representations of information, data, or knowledge intended to present information quickly and clearly. They can improve cognition by utilizing graphics to enhance the human visual system's ability to see patterns and trends." \footnote{https://en.wikipedia.org/wiki/Infographic - 31.03.2022}

\paragraph{Technologies}
The project will be built using mainly JavaScript and HTML, with a bit of CSS. This allows to easily adapt the results into all kinds of web based applications, without compatibility issues which could arise by using a framework. Additionally we use the D3.js library. "[It] is a JavaScript library for manipulating documents based on data. D3 helps you bring data to life using HTML, SVG, and CSS." \footnote{https://d3js.org/ - 31.03.2022}

Versioning control is done using a git repository. For easy access, the repository is hosted on github. The raw datasets are not provided in git directly, as the files are too big. Their sources are provided though.

To faster comprehend the huge amounts of available raw data, Python is used. It helps quickly parsing the data and identifying interesting aspects. It is also used to extract the relevant data we want to use.

\paragraph{Diagrams}
Diagrams or charts are visual representations of data. They help to understand and comprehend big amounts of data. Different types of diagrams can show different types of data. Quantitative data might need other representation than categorical data. The most common ways to encode data are using the x and y positions in a graph, areas and color.

Digital diagrams can also use transitions and animations to show relations between different aspects, show changes over time, or to simple be more visually appealing to potential readers.

\paragraph{Data}
Data is produced everywhere. Most of it is created in a commercial setting. But there are many public data sources available as well. Some is open source or available through api requests. Yet most of the available data on current events is highly individual and not machine readable. This thesis only uses data which is publicly available and machine readable. As this thesis focuses on the technical implementation and to show the possibilities of D3, we do neither need nor claim the data to be correct or unbiased. The data used in this thesis comes from the UNHCR, the UN Refugee Agency.

\section{Selection process}
Selecting the data and diagrams to show is not an easy task. When selecting the data used it is important to find data with different attribute types. There should be data using quantitative, but also ordered and categorical attributes. This allows for a wider range of possible diagrams. When selecting the type of diagram, it is important to pay attention to what diagram can represent which types of data attributes well. Usually it is also worth paying attention to how accurately the presented data is perceived. This depends on the chosen shapes, colors and diagram types. As we are trying to create a showcase for the possibilities of D3, we do not need to consider the best didactic choices. Yet the chosen combinations of data and diagrams should still be reasonable and realistically usable.

\section{Creating a showcase}
The showcase is centered around the refugees of the current Ukraine conflict. For any chosen dataset, there should be the possibility to manipulate the data table within the browser on one side of the screen. On the other side, we can scroll through several types of diagrams, visualizing the same dataset. Allowing the user to manipulate the data or time frame in the browser, will allow for full use of D3's adaptability to data changes. Creating different diagrams for the same set of data also allows easy comparison of the implementation of different diagrams. Even though the showcase is trying to compare diagrams with each other, it can easily be imagined that single diagrams are adopted by news agencies or state websites to provide an overview of the current situation. Therefore the showcase should be build in a modular way, which makes it easy to reuse the single diagrams.

\section{Expectations}
Choosing interesting and representative data for the showcase will be a challenge. Every person has different interests and could find different aspects interesting. Furthermore it is very difficult finding machine readable data about the ongoing conflict. Yet the amount of data available on the country is enormous. Looking through all the available data is a huge time investment. Therefore it is important not to get carried away in wanting to present more and more.
I think the possibilities of creating the showcase are quite unlimited. It will be an interesting journey going through and refining the different aspects. But I do think the benefits of having such a showcase can be quite strong. Besides acting as a baseline and possible starting point for others in understanding and working with D3, a lot, if not most, people, including myself, only have a vague idea about the Ukraine as a country. By creating this showcase I hope to provide context about the country we so commonly see in the news today. It might even motivate people to increase or start their active involvement against the war.


%%%
%%% end main document
%%%
%%%%%%%%%%%%%%%%%%%%%%%%%%%%%%%%%%%%%%%%%%%%%%%%%%%%%%%%%%%%%%%%%%%%%%%%%%%%%%%%

% \appendix  %% include it, if something (bibliography, index, ...) follows below

%%%%%%%%%%%%%%%%%%%%%%%%%%%%%%%%%%%%%%%%%%%%%%%%%%%%%%%%%%%%%%%%%%%%%%%%%%%%%%%%
%%%
%%% bibliography
%%%
%%% available styles: abbrv, acm, alpha, apalike, ieeetr, plain, siam, unsrt
%%%
% \bibliographystyle{plain}

%%% name of the bibliography file without .bib
%%% e.g.: literature.bib -> \bibliography{literature}
% \bibliography{FIXXME}

\end{document}
%%% }}}
%%% END OF FILE
%%%%%%%%%%%%%%%%%%%%%%%%%%%%%%%%%%%%%%%%%%%%%%%%%%%%%%%%%%%%%%%%%%%%%%%%%%%%%%%%
%%% Notice!
%%% This file uses the outline-mode of emacs and the foldmethod of Vim.
%%% Press 'zi' to unfold the file in Vim.
%%% See ':help folding' for more information.
%%%%%%%%%%%%%%%%%%%%%%%%%%%%%%%%%%%%%%%%%%%%%%%%%%%%%%%%%%%%%%%%%%%%%%%%%%%%%%%%
%% Local Variables:
%% mode: outline-minor
%% OPToutline-regexp: "%% .*"
%% OPTeval: (hide-body)
%% emerge-set-combine-versions-template: "%a\n%b\n"
%% End:
%% vim:foldmethod=marker
