%%% File-Information {{{
%%% Filename: template_bericht.tex
%%% Purpose: lab report, technical report, project report
%%% Time-stamp: <2004-06-30 18:19:32 mp>
%%% Authors: The LaTeX@TUG-Team [http://latex.tugraz.at/]:
%%%          Karl Voit (vk), Michael Prokop (mp), Stefan Sollerer (ss)
%%% History:
%%%   20050914 (ss) correction of "backref=true" to "backref" due to hyperref documentation
%%%   20040630 (mp) added comments to foldmethod at end of file
%%%   20040625 (vk,ss) initial version
%%%
%%% Notes:
%%% Adapted for my bachelor thesis exposé
%%%
%%%
%%% }}}
%%%%%%%%%%%%%%%%%%%%%%%%%%%%%%%%%%%%%%%%%%%%%%%%%%%%%%%%%%%%%%%%%%%%%%%%%%%%%%%%
%%% main document {{{

\documentclass[
a4paper,     %% defines the paper size: a4paper (default), a5paper, letterpaper, ...
% landscape,   %% sets the orientation to landscape
% twoside,     %% changes to a two-page-layout (alternatively: oneside)
% twocolumn,   %% changes to a two-column-layout
% headsepline, %% add a horizontal line below the column title
% footsepline, %% add a horizontal line above the page footer
% titlepage,   %% only the titlepage (using titlepage-environment) appears on the first page (alternatively: notitlepage)
% parskip,     %% insert an empty line between two paragraphs (alternatively: halfparskip, ...)
% leqno,       %% equation numbers left (instead of right)
% fleqn,       %% equation left-justified (instead of centered)
% tablecaptionabove, %% captions of tables are above the tables (alternatively: tablecaptionbelow)
% draft,       %% produce only a draft version (mark lines that need manual edition and don't show graphics)
% 10pt         %% set default font size to 10 point
% 11pt         %% set default font size to 11 point
12pt         %% set default font size to 12 point
]{scrartcl}  %% article, see KOMA documentation (scrguide.dvi)



%%%%%%%%%%%%%%%%%%%%%%%%%%%%%%%%%%%%%%%%%%%%%%%%%%%%%%%%%%%%%%%%%%%%%%%%%%%%%%%%
%%%
%%% packages
%%%

\usepackage{fancyhdr}

%%%
%%% encoding and language set
%%%

%%% ngerman: language set to new-german
%\usepackage{ngerman}

%%% babel: language set (can cause some conflicts with package ngerman)
%%%        use it only for multi-language documents or non-german ones
\usepackage[english]{babel}

%%% inputenc: coding of german special characters
\usepackage[utf8]{inputenc}

%%% fontenc, ae, aecompl: coding of characters in PDF documents
\usepackage[T1]{fontenc}
\usepackage{ae,aecompl}

%%%
%%% technical packages
%%%

%%% amsmath, amssymb, amstext: support for mathematics
%\usepackage{amsmath,amssymb,amstext}

%%% psfrag: replace PostScript fonts
\usepackage{psfrag}

%%% listings: include programming code
%\usepackage{listings}

%%% units: technical units
%\usepackage{units}

%%%
%%% layout
%%%

%%% scrpage2: KOMA heading and footer
%%% Note: if you don't use this package, please remove 
%%%       \pagestyle{scrheadings} and corresponding settings
%%%       below too.
%%% \usepackage[automark]{scrpage2}


%%%
%%% PDF
%%%

\usepackage{ifpdf}

%%% Should be LAST usepackage-call!
%%% For docu on that, see reference on package ``hyperref''
\ifpdf%   (definitions for using pdflatex instead of latex)

  %%% graphicx: support for graphics
  \usepackage[pdftex]{graphicx}

  \pdfcompresslevel=9

  %%% hyperref (hyperlinks in PDF): for more options or more detailed
  %%%          explanations, see the documentation of the hyperref-package
  \usepackage[%
    %%% general options
    pdftex=true,      %% sets up hyperref for use with the pdftex program
    %plainpages=false, %% set it to false, if pdflatex complains: ``destination with same identifier already exists''
    %
    %%% extension options
    backref,      %% adds a backlink text to the end of each item in the bibliography
    pagebackref=false, %% if true, creates backward references as a list of page numbers in the bibliography
    colorlinks=false,   %% turn on colored links (true is better for on-screen reading, false is better for printout versions)
    %
    %%% PDF-specific display options
    bookmarks=true,          %% if true, generate PDF bookmarks (requires two passes of pdflatex)
    bookmarksopen=false,     %% if true, show all PDF bookmarks expanded
    bookmarksnumbered=false, %% if true, add the section numbers to the bookmarks
    %pdfstartpage={1},        %% determines, on which page the PDF file is opened
    pdfpagemode=None         %% None, UseOutlines (=show bookmarks), UseThumbs (show thumbnails), FullScreen
  ]{hyperref}


  %%% provide all graphics (also) in this format, so you don't have
  %%% to add the file extensions to the \includegraphics-command
  %%% and/or you don't have to distinguish between generating
  %%% dvi/ps (through latex) and pdf (through pdflatex)
  \DeclareGraphicsExtensions{.pdf}

\else %else   (definitions for using latex instead of pdflatex)

  \usepackage[dvips]{graphicx}

  \DeclareGraphicsExtensions{.eps}

  \usepackage[%
    dvips,           %% sets up hyperref for use with the dvips driver
    colorlinks=false %% better for printout version; almost every hyperref-extension is eliminated by using dvips
  ]{hyperref}

\fi


%%% sets the PDF-Information options
%%% (see fields in Acrobat Reader: ``File -> Document properties -> Summary'')
%%% Note: this method is better than as options of the hyperref-package (options are expanded correctly)
\hypersetup{
  pdftitle={}, %%
  pdfauthor={}, %%
  pdfsubject={}, %%
  pdfcreator={Accomplished with LaTeX2e and pdfLaTeX with hyperref-package.}, %% 
  pdfproducer={}, %%
  pdfkeywords={} %%
}


%%% rename table of contants
%%% replace    english  with the language you use
\addto\captionsenglish{
  \renewcommand{\contentsname}
    {Table of contents} % Title goes in here
}


%%%%%%%%%%%%%%%%%%%%%%%%%%%%%%%%%%%%%%%%%%%%%%%%%%%%%%%%%%%%%%%%%%%%%%%%%%%%%%%%
%%%
%%% user defined commands
%%%

%%% \mygraphics{}{}{}
%% usage:   \mygraphics{width}{filename_without_extension}{caption}
%% example: \mygraphics{0.7\textwidth}{rolling_grandma}{This is my grandmother on inlinescates}
%% requires: package graphicx
%% provides: including centered pictures/graphics with a boldfaced caption below
%% 
\newcommand{\mygraphics}[3]{
  \begin{center}
    \includegraphics[width=#1, keepaspectratio=true]{#2} \\
    \textbf{#3}
  \end{center}
}

%%%%%%%%%%%%%%%%%%%%%%%%%%%%%%%%%%%%%%%%%%%%%%%%%%%%%%%%%%%%%%%%%%%%%%%%%%%%%%%%
%%%
%%% define the titlepage
%%%

% \subject{}   %% subject which appears above titlehead
% \titlehead{} %% special heading for the titlepage

%%%% title
%\title{d3.js and its possibilities in infographics\\
%\large Creating a diagram showcase using ukrainian refugee data\\
%of the ongoing armed conflict}
%
%%%% author(s)
%\author{Luis Rothenh\"ausler (20202459)}
%
%%%% date
%\date{Brandenburg, \today{}}
%
%\publishers{Professor Julia Schnitzer}

\newcommand{\THBTitle}[9]{

  \thispagestyle{empty}
  \vspace*{\stretch{1}}
  {\parindent0cm
  \rule{\linewidth}{.7ex}}
  \begin{flushright}
    \vspace*{\stretch{1}}
    \sffamily\bfseries\Huge
    #1\\
    \vspace*{\stretch{1}}
    \sffamily\bfseries\large
    #2\\
    \vspace*{\stretch{1}}
    \sffamily\bfseries\small
    #3
  \end{flushright}
  \rule{\linewidth}{.7ex}

  \vspace*{\stretch{1}}
  \begin{center}
    \includegraphics[width=2in]{figs/2015_10_05_THB_FB-IM_Logo_RGB} \\
    \vspace*{\stretch{1}}
    \Large  Bachelorarbeit - Exposé\\

    \vspace*{\stretch{2}}
   \large Fachbereich Informatik\\
    \large und Medien\\
    \large Technische Hochschule Brandenburg\\
    \vspace*{\stretch{1}}
    \large Betreuer:  #8 \\[1mm]
    \large 2. Betreuer:  #9 \\[1mm]
    
    \vspace*{\stretch{1}}
    \large Brandenburg, den #7 \\
        \vspace*{\stretch{0.25}}

    Bearbeitungszeit: dd.mm.yyyy - dd.mm.yyyy % Die Bearbeitungszeit der Seminar-/ Abschlussarbeit ist hier einzutragen.

  \end{center}
}

% \thanks{} %% use it instead of footnotes (only on titlepage)

% \dedication{} %% generates a dedication-page after titlepage


%%% uncomment following lines, if you want to:
%%% reuse the maketitle-entries for hyperref-setup
%\newcommand\org@maketitle{}
%\let\org@maketitle\maketitle
%\def\maketitle{%
%  \hypersetup{
%    pdftitle={\@title},
%    pdfauthor={\@author}
%    pdfsubject={\@subject}
%  }%
%  \org@maketitle
%}


%%%%%%%%%%%%%%%%%%%%%%%%%%%%%%%%%%%%%%%%%%%%%%%%%%%%%%%%%%%%%%%%%%%%%%%%%%%%%%%%
%%%
%%% set heading and footer
%%%

%%% scrheadings default: 
%%%      footer - middle: page number
%%% \pagestyle{scrheadings}

%%% user specific
%%% usage:
%%% \position[heading/footer for the titlepage]{heading/footer for the rest of the document}

%%% heading - left
% \ihead[]{}

%%% heading - center
% \chead[]{}

%%% heading - right
% \ohead[]{}

%%% footer - left
% \ifoot[]{}

%%% footer - center
% \cfoot[]{}

%%% footer - right
% \ofoot[]{}



%%%%%%%%%%%%%%%%%%%%%%%%%%%%%%%%%%%%%%%%%%%%%%%%%%%%%%%%%%%%%%%%%%%%%%%%%%%%%%%%
%%%
%%% begin document
%%%

\begin{document}

\THBTitle
      {d3.js and its possibilities in infographics\\
      \large Creating a diagram showcase using ukrainian refugee data\\
      of the ongoing armed conflict}        % Titel der Arbeit
      {Luis Rothenh\"ausler}                        % Vor- und Nachname des Autors
      {20202459}
      
      {Technische Hochschule Brandenburg}  % Name der Fakultaet
      {Brandenburg 2022}                          % Ort und Jahr der Erstellung
      {14.05.2022}                              % Tag der Abgabe
      {Prof. Julia Schnitzer}               % Name des Erstgutachters
      {Hier k\"onnte Ihr Name stehen}                          % Name des Zweitgutachters

% \pagenumbering{roman} %% small roman page numbers

%%% include the title
% \thispagestyle{empty}  %% no header/footer (only) on this page
%\maketitle

%%% start a new page and display the table of contents
\newpage
\setcounter{page}{1}
\tableofcontents

%%% start a new page and display the list of figures
% \newpage
% \listoffigures

%%% start a new page and display the list of tables
% \newpage
% \listoftables

%%% display the main document on a new page 
% \newpage

% \pagenumbering{arabic} %% normal page numbers (include it, if roman was used above)

%%%%%%%%%%%%%%%%%%%%%%%%%%%%%%%%%%%%%%%%%%%%%%%%%%%%%%%%%%%%%%%%%%%%%%%%%%%%%%%%
%%%
%%% begin main document
%%% structure: \section \subsection \subsubsection \paragraph \subparagraph
%%%
\newpage

\section{Introduction}
The postmodern world produces huge amounts of data every second. Analyzing this data leads to better-informed decision-making in every sector. Yet the wast amounts of created data is often hard to comprehend with the human mind. Infographics is about finding ways to represent this data in visually appealing, yet easily understandable visual representations. Doing this quickly and always up to date can be crucial. There are many tools available to help with the creating of infographics. Some of these tools are software based, others require programming. This thesis will be a deep dive into the possibilities of one of these tools, the 'd3.js'(D3) library for JavaScript. To show its capabilities we will use D3 to create a showcase containing several different graphics.

The following sections describe what the goal of this thesis is, as well as why they should be pursued and how they will be achieved.

\section{Basics}
D3 stands for "Data-Driven Documents" and "[...] is a JavaScript library for manipulating documents based on data. D3 helps you bring data to life using HTML, SVG, and CSS." \cite{d3js}
Since the first notation of the development of D3 in 2011, the library has seen continuos development. During this time many different resources and usage examples have been created for D3. Yet a lot of these openly available examples are created in online environments with D3 preloaded, are lacking documentation, use different styles of JavaScript or do not react to changing data.

\section{Goals}
The main result of this thesis will be the creation of a showcase of diagrams created in D3. And there are several questions that will be answered in the process. What are the strengths and weaknesses of D3? How easy is D3 to use? How hard is it to initially get into? When is D3 a viable option for creating graphics? 

Furthermore there are certain requirements the showcase should fulfill. Each diagram should be well documented and explained, be easily adaptable to ones personal use and use some aspect of D3 which is not used by the other diagrams. They should also be easily comparable to each other, to allow fast comprehension and understanding of the differences in the creation process. Furthermore all diagrams should be able to react to data changes, so they can continuously update as the data changes.

\section{How}
To achieve our goals, we will set certain limitations. Besides using a consistent code style and proper documentation, each diagram should be fully independent of the other graphics. This will allow easier adaptation and an easier comparison in effort between the graphics. To further simplify the comparison, we will use a limited number of datasets. Furthermore the diagrams should be implemented using only JavaScript, HTML, CSS and the D3 library, and not rely on a specific framework. This makes them lightweight and easier to adapt them to all kinds of web based projects. To mimic changing data, the showcase has an option to manipulate the used data-sets.

To make sure we actually make proper use of the D3 library, and be able to evaluate the strengths and possibilities of D3, we will make use of as many different sections of the API\cite{d3_api} as possible.

\section{Data}
Data is produced everywhere. Most of it is created in a commercial setting. But there are many public data sources available as well. Some is open source or available through api requests. Yet most of the available data on current events is highly individual and not machine readable. This thesis only uses data which is publicly available and machine readable. As this thesis focuses on the technical implementation and to show the possibilities of D3, we do neither need nor claim the data to be correct or unbiased.

Data can be categorized into two main categories. Qualitative and quantitative data. Each category has two subtypes. Qualitative dta can be either nominal and ordinal. Whilst nominal data does not have a natural order, ordinal data does. Quantitative data can be either discreet or continuos. Different types of data can be shown differently in diagrams. In our implementations we will use data of all four kinds. Our first dataset contains the refugees in absolute people(discreet) per country(nominal). we can also express this in refugees in percent(continuos) per country. The second dataset is about the total refugee count(discreet) over time(ordinal).

All data comes from the UNHCR, the UN Refugee Agency\cite{unhcr}.

\section{Diagrams}
When selecting the type of diagram, it is important to pay attention to what diagram can represent which types of data well. Usually it is also worth paying attention to how accurately the presented data is perceived. This depends on the chosen shapes, colors and diagram types\cite{heer2010crowdsourcing}. It can also differ depending on the types of data which are presented\cite{mackinlay1986automating}. As we are trying to create a showcase for the possibilities of D3, we will not need to consider the best didactic choices. Yet the chosen combinations of data and diagrams should still be reasonable and realistically usable, while using a wide range of D3's possibilities.

\section{Expectations}
I think the possibilities of creating the showcase are quite unlimited. It will be an interesting journey going through and refining the different aspects. But I do think the benefits of having such a showcase can be quite strong. It can act as a baseline and possible starting point for others in understanding and working with D3. 

Furthermore I think it will be interesting to see in which cases using D3 can be reasonably justified. As the visualization tools of other software, i.e. Excel, are quite strong, I assume that there are many cases where it can be easier and faster to not use D3. 

\section{Preliminary Structure}
This is the preliminary structure of the actual bachelors thesis:

\begin{itemize}
  \item German Abstract
  \item Introduction
  \item Basics
  \begin{itemize}
    \item Infographics
    \item D3.js
    \item Data
    \item Diagrams
  \end{itemize}
  \item Implementation
  \begin{itemize}
    \item Diagrams
    \item Showcase
  \end{itemize}
  \item Conclusion
  \item Sources
  \item Appendix
\end{itemize}

\newpage

%%%
%%% end main document
%%%
%%%%%%%%%%%%%%%%%%%%%%%%%%%%%%%%%%%%%%%%%%%%%%%%%%%%%%%%%%%%%%%%%%%%%%%%%%%%%%%%

\appendix  %% include it, if something (bibliography, index, ...) follows below

%%%%%%%%%%%%%%%%%%%%%%%%%%%%%%%%%%%%%%%%%%%%%%%%%%%%%%%%%%%%%%%%%%%%%%%%%%%%%%%%
%%%
%%% bibliography
%%%
%%% available styles: abbrv, acm, alpha, apalike, ieeetr, plain, siam, unsrt
%%%
\bibliographystyle{IEEEtran}

%%% name of the bibliography file without .bib
%%% e.g.: literature.bib -> \bibliography{literature}
\bibliography{sources}

\end{document}
%%% }}}
%%% END OF FILE
%%%%%%%%%%%%%%%%%%%%%%%%%%%%%%%%%%%%%%%%%%%%%%%%%%%%%%%%%%%%%%%%%%%%%%%%%%%%%%%%
%%% Notice!
%%% This file uses the outline-mode of emacs and the foldmethod of Vim.
%%% Press 'zi' to unfold the file in Vim.
%%% See ':help folding' for more information.
%%%%%%%%%%%%%%%%%%%%%%%%%%%%%%%%%%%%%%%%%%%%%%%%%%%%%%%%%%%%%%%%%%%%%%%%%%%%%%%%
%% Local Variables:
%% mode: outline-minor
%% OPToutline-regexp: "%% .*"
%% OPTeval: (hide-body)
%% emerge-set-combine-versions-template: "%a\n%b\n"
%% End:
%% vim:foldmethod=marker
