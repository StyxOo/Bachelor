% This is the german abstract which is required, for some reason.
\chapter*{Abstract - German} % Chapter with * doesn't number
%Englische Arbeiten brauchen eine Zusammenfassung auf Deutsch. Mal abgesehen davon, dass wenn die Zusammenfassung interessant ist man ohne English eh nicht weiter kommt...

%Die moderne Welt produziert täglich riesige Mengen an Daten. Diese zu verstehen und auszuwerten kann in allen Bereichen Grundlage für Entscheidungen und Entwicklung sein. Da es oft schwierig ist große Datensätze zu überblicken und Schlussfolgerungen daraus zu ziehen, wird Datenvisualisierung genutzt Datensätze begreifbar und überschaubar zu machen. Die Ergebnisse von Datenvisualisierung lassen sich überall im Alltag finden. Zum Beispiel als Diagramme in Zeitungen und Nachrichten, sowie die Darstellung von Arbeitsbläufen und Zusanmenhängen unterschiedlicher Faktoren.

%Es gibt eine vielzahl an Werkzeugen die zur Datenvisualisierung genutzt werden können. Während manche dieser Werkzeuge, wie Excel und SPSS, eine Grafische Oberfläche bieten, nutzen andere, wie R und Matplotlib, einen code basierten Ansatz. Die Entscheidung welches der Werkzeuge für die eigenen Anforderungen am besten geeignet ist, ist jedoch nicht immer einfach. Besonders, da die Möglichkeiten, so wie die Vor- und Nachteile der eizelnen Werkzeuge für jemand ohne vorhandene Erfahrung nicht einfach zu überblicken sind. 

Diese Arbeit gibt einen Einblick, sowie eine Einschätzung eines dieser der code basierten Werkzeuge zur Datenvisualierung, die d3js JavaScript library. Es werden nicht nur das Potential, sondern auch die Vor- und Nachteile evaluiert. Weiterhin wird eine Einschätzung gegeben, wann der Einsatz dieses Werkzeugen sinnvoll ist. Um diese Einschätzungen treffen zu können, werden einige Diagramme erstellt. Dafür werden zunächst die nötigen Grundlagen erklärt. Dies umfasst sowohl Daten und Datentypen, als auch Diagramm und deren Aufbau. Abschließend werden die erstellten Diagramme, sowie d3js im Diskussionsteil ausgewertet.

Da d3js in reinem JavaScript implementiert ist, lässt es sich Problemlos mit anderen Frameworks kombinieren. Der Fokus von d3js liegt auf der schnellen und einfachen Manipulation von Elementen im Document Object Model. d3js ist jedoch kein Werkzeug mit dem in wenigen Zeilen Quellcode ganze Diagramme erstellt werden können. Vielmehr müssen sowohl die einzelnen Elemente eines Diagrammes, wie auch deren Position und Aussehen händisch definiert werden. Das ermöglicht ein höchstmaß an Kontrolle über das Aussehen und Verhalten der Diagramme. Es ist jedoch auch Zeitintesiv und erlaubt für die Implementation hüchst unübersichtilicher Diagramme. Gepaart mit der hohen initialen Lernkurve, kann die Nutzung von d3js nur in speziellen Fällen, welche die volle Kontrolle über Aussehen und Verhalten benötigen, empfohlen werden.