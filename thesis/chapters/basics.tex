\chapter{Basics}
Everything necessary to understand the implementation as well as anything which is done beforehand, will come up here

In the following, all concepts, technologies and required backgrounds for understanding this thesis are explained. 


\section{Data}
Well talk about data a bit. Where does it come from? How is it structured? What kind of attributes? What even are attributes?

Since ancient times, humans have recorded data. Recording the ins and outs of available resources was one of the driving factors behind the conceptualization of writing. (TODO: Check sources of the beer brewing video series)
With the introduction of computers the amounts of gathered data have grown drastically. Vast amounts of data are gathered across all aspects of life.

\subsection{Types of Data}
Even though data comes from a huge variety of sources and can express a plethora of things, there are only four different types of data. They are split into equal pairs. Two types of categorical data and two types of numerical data. In the following each of the types of data will be explained.

\subsubsection{Categorical}
What is categorical data? Nominal and ordinal data

Categorical data is information collected in groups. It is often of descriptive nature. Whilst the values can be represented in numbers, they do not allow for arithmetic operations.
There are two types of categorical data. Nominal and ordinal data.

\paragraph{Nominal}
data is mostly descriptive in nature. They are independent and have no inherit order. Examples are 'Country', 'Color', 'Brand'.

\paragraph{Ordinal}
data is mostly similar to nominal data. Yet the data does have some sort of internal order. For example different dates each describe a day, but one day also comes after another.

\subsubsection{Numeric}
What is numeric data? Continuous and ???

\subsection{Datasets}
What are our datasets about? Where do they come from?

\subsubsection{Preprocessing}
What is done in preprocessing? Python script which removes all excess / maybe do that in JS as well..?

\subsubsection{Data Types}
Which data types can be found in our data-sets? Where?


\section{Data Visualization}
What is it? Where is it? Why is it important?


\section{Diagrams}
What diagrams exists? Which are the most common? What possibilities do they offer for encoding data? Which considerations for readability? Why do some diagrams not make as much sense? Which considerations where made for fulfilling the showcase requirements?

\subsection{Refugees per country}
Which ones did I choose? Why? Which data attributes do they encode?

\subsubsection{Bar chart}
What is it? Why is it here?

\subsubsection{Pie chart}
What is it? Why is it here?

\subsubsection{Tree map}
What is it? Why is it here?

\subsubsection{Sankey}
What is it? Why is it here?

\subsection{Refugees over time}
Which ones did I choose? Why? Which data attributes do they encode?

\subsubsection{Area graph}
What is it? Why is it here?

\subsubsection{Circle graph}
What is it? Why is it here?


\section{D3.js}
This is all about d3. What is it? Where does it come from? What is it used for? Who uses it? Why should it be used? How does it work? Enter, update and exit pattern. Something about the modular structure of D3 as well. Might be worth mentioning "observables" as well.

\subsection{What is it?}
General definition. What is it, where does it come from, what is it for?

\subsection{How does it work?}
General functioning of D3.

\subsubsection{General Update Pattern}
What is it? What can it do? Describe data joins and dom element links.

\subsubsection{Modules}
The way D3 is split up into modules, the core package and what kind of extensions are there.
