\chapter{Discussion}\label{sec:discussion}
The discussion evaluates the results of this thesis. Therefore it is split into two parts. The first part, section \ref{sec:diagrams-dis}, is about the diagrams and comparing them to each other. Section \ref{sec:d3-dis} contains the second part of the discussion and tries to answer the initial questions about the potential of D3 in data visualization.

\section{Diagrams}\label{sec:diagrams-dis}
The first four diagrams show the data about the refugees by country crossed into. While the underlying data is the same, the resulting diagrams are quite different from each other.
The donut chart and tree map make it immediately obvious, that almost half of all refugees went to Poland. While this is hard to guess from the bar chart without a ruler or calculator, the vertical spacing of the nodes in the Sankey diagram also does not make this obvious.
Only the donut chart and the Sankey diagram allow the viewer to immediately see the total amount of refugees. While the bar chart and tree map also contain this information, they would require the viewer to calculate it themselves, defeating the purpose of data visualization.
The Sankey diagram and bar chart directly show all the numbers of refugees per country. This information can be found in the donut chart and tree map as well, but only by using the mouse to inspect the diagrams more closely. This can be a struggle when trying to get more precise data about the refugees which went to Belarus, as their amount is so small, that these two diagrams hardly show Belarus at all.
While the shown legend is universally true, both the Sankey diagram and the bar chart also directly show the names of the countries, making comprehension of the data more easy.
The Sankey diagram is also the only diagram which conveys the flow from the Ukraine to the other countries. The other diagrams only achieve this through context, not in their actual graphic.

When data is updated, all four diagrams smoothly transition to their new states. This makes the changes quite easy to follow and pleasant to look at. The tree map can sometimes be a bit confusing, as the different rectangles can shift across each other.

The two diagrams showing the cumulative refugees per day, are quite different from each other. While the circle graph actually only ever shows the value for a single day, the area graph shows the whole extend of the data with an extra indicator for the current day.
Due to the scaling of the radius in the circle diagram it can also be quite hard to read the number of refugees from the legend, especially for smaller values.
The area graph does a much better job of not only consistently showing not the number of refugees on the selected day, but also the development of the refugee count over time. Therefore it struggles with accurately showing the current date. This is easier in the circle chart, as it is simply displayed at the bottom.

When data is updated, the circle graph can be completely unaffected. Only when the data for the currently selected day is changed, or the size legend adjusts itself, changes can be seen. Because of the way that the functions used to create the line and area of the area graph work, the updates of the area graph are only easy to follow when values are adjusted. When days are added and removed it is hard to follow along, due to how the line and area are created by D3.


\section{D3}\label{sec:d3-dis}
The tree initial questions presented which this thesis tries to answer were: What is the potential of D3 in data visualization? What are the advantages and disadvantages of using D3? When should D3 be used? All three of these questions will be evaluated and discussed separately in the following discussion, beginning with the question of what D3s potential is. 

Looking at the created showcase, D3 can obviously be used to create many different types of diagrams. Looking at the examples found online, D3 has their own showcase of projects, makes this even more apparent. From simple bar and pie charts, over visualizing hierarchical data using tree maps or Sankey diagrams, like the ones created for this thesis, all the way to map based diagrams using various projections\cite{davies}, physics enabled bubble graphs\cite{carter_2012} and pseudo 3D animations\cite{davies_sphere}. Of course D3 can also be used to created animations which do not necessarily serve data visualization purposes, like the tadpoles example\cite{bostock_2020}. Due to D3's low level approach, fast speed, and the general update pattern, D3 can be used to create all visualizations one can imagine and have them react to data changes in real time. As D3 is built around simple DOM manipulation, it can also be used for other use cases. One can create HTML tables and populate them with data. One can make small animations to add visually appealing aspects to a website. Or maybe one can adapt the scales for their own needs of converting data. But what are the advantages an disadvantages of D3?

When working in a web environment, it is always easy to start using D3. Its independence from any framework and its pure JavaScript implementation, make it possible to include D3 in any web-based project. On the other hand, if one is not already working on a web-based project but still wants to use D3, one has to deal with all the additional overhead of working with a web technology stack. Whilst importing the D3 library is really easy, the initial learning curve is everything but easy. Without first internalizing the core concepts of D3 and SVG, it is impossible to make any kind of visualization. While this is true for any framework or technology, the concept of the data joins and the general update pattern seem especially abstract. Yet it is crucial to understand it when a diagram is supposed to react to data changes.
While learning the basics of D3 is quite a big hurdle, it can be broken down by first creating only static diagrams. This can be achieved without a deeper understanding of the general update pattern. While D3s low-level approach is cumbersome to comprehend initially, it actually allows D3 to be very flexible.
In addition, while there are a lot of examples and tutorials, they are often too complex to understand as a beginner, or use varying versions of D3 or JavaScript styles. Besides some functionalities of D3 being obsolete in newer versions, the different styles of JavaScript can be additionally confusing when the developer is not familiar with the evolution of JavaScript.
Once one understands how D3 works, it can be quite fast to create basic diagrams. Once one properly understands the general update pattern and selections, it is also not too difficult to react to data changes. Yet trying to animate diagrams can become tricky. It depends on the elements which are used and which attributes need to be animated. As all elements have to be manually specified, one has full control over the appearance and behavior of the diagrams. This allows D3 to adapt to any existing style guides, yet is also very time consuming. It also allows creating diagrams with a high lie-factor or a bad data-ink ratio. This risk can be mitigated using a more high-level library. High-level libraries also allow for the automatic creation of legends, instead of having to create them manually as was done in this thesis.

Another concern might be the performance of D3 when dealing with many marks in a diagram. Therefore the limits of smooth animation of the bar chart and the tadpoles example\cite{bostock_2020} were briefly tested. Of course these results vary from device to device. In this case, it was possible to smoothly animate up to approximately 1200-1300 tadpoles. The bar chart is harder to evaluate. The singular bars approach, and partially pass, the limits of a singular pixel in thickness, and therefore their visibility on the screen, with this number of entries in the diagram. While this thesis does not provide a proper performance test and comparison with other visualization libraries performances, it can be safely assumed that performance will not be an issue for most diagrams.

So when should one use D3? This depends on the task at hand. If the goal is to create some diagrams as a one time job, D3 is unnecessarily complex. Tools like Excel can easily excel here. Even when working with a web project, D3 is probably too complex. Most developers will find all necessarily functionality using libraries like Chart.js\cite{chartjs} or the Plotly JavaScript library, which is actually built on top of D3 to offer a more high level approach to data visualization. Both these open-source libraries allow creating some of the most common diagrams without having to learn all the quirks of D3.
Yet if someone is to create a custom visualization or wants more fine control over all aspects of a diagram, D3 can handle it.
Nevertheless, due to the high initial learning curve getting into D3 can only be recommended when the creating custom visualizations, which high level libraries can not provide.
