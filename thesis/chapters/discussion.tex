\chapter{Discussion}

As there are three main question this thesis aims to answer, each question will be discussed separately. Beginning with the question of what D3 can do. Looking at the created showcase, D3 can obviously be used to create many different types of diagrams. Looking at the examples found online, D3 has their own showcase of projects, makes this even more apparent. From simple bar and pie charts, over visualizing hierarchical data using tree-maps or sankey-diagrams, like the ones created for this thesis, all the way to map based diagrams using various projections\cite{davies}, physics enabled bubble-graphs\cite{carter_2012} and pseudo 3D animations\cite{davies_sphere}. Of course D3 can also be used to created animations which do not necessarily serve data visualization purposes, like the tadpoles example\cite{bostock_2020}. Due to D3's low level approach, fast speed, and the general update pattern, D3 can be used to create all visualizations one can imagine and have them react to data changes in real time. As D3 is built around simple DOM manipulation, it could also be used for other aspects. One could draw charts and populate them with data. One can make small animations to add visually appealing aspects to a website. Or maybe one can adapt the scales for their own needs of converting data. But what does D3 excel at and where does it struggle?

When working in a web environment, it is always easy to start using D3 as well. Its independence from any framework and implementation only in JavaScript, makes it possible to include D3 in any web-based project. On the other hand, if one is not already working on a web-based project but still wants to use D3, one has to deal with all the additional overhead of hosting a, at least, local web-server before being able to use D3. Whilst importing the D3 library is really easy, the initial learning curve is everything but easy. Without first internalizing the core concepts of D3 and SVG, it is impossible to make any kind of visualization. Especially when it is supposed to react to data changes. Learning D3 is also a double edged sword. Whilst there are a lot of examples and tutorials, they are often to complex to understand as a beginner, or use varying versions of D3 or JavaScript styles. Besides some functionalities of D3 being obsolete in newer versions, the different styles of JavaScript can be additionally confusing when unfamiliar. Once one understands how D3 works, it can be quite fast to create basic diagrams. From here it is quite easy to make the diagrams react to data changes, once one properly understands the general update pattern and selections. Yet trying to animate diagrams can be tricky. It depends on the elements which are used and which attributes need to be animated. As all elements have to manually specified, one has full control over the appearance and behavior of the diagrams. Yet having to specify all aspects so precisely is also quite time consuming. Another concern might be the performance of D3 when dealing with many marks in a diagram. Therefore the limits of smooth animation of the bar-chart and the tadpoles example\cite{bostock_2020} were briefly tested. Of course these results vary from device to device. In this case, it was possible to smoothly animate up to approximately 1200-1300 tadpoles. The bar-chart is harder to evaluate. The singular bars approach, and partially pass, the limits of a singular pixel in thickness and therefore their visibility with this number of entries in the diagram. Concluding this rudimentary test, D3 should not reach its limitations in animating diagrams smoothly, as long as the data to represent does not reside within the realm of big-data.

So when should one use D3? This depends on the task at hand. If the goal is to create some diagrams as a one time job, D3 is unnecessarily complex. Tools like Excel can easily excel here. Even when working with a web project, the resulting images can easily be imported here. When working with data which changes over time, but not in a web project, it is probably not worth introducing the additional overhead. If one is already working on a web project and with data changing over time, D3 should be seriously considered. Due to the easy integration with any framework and possibility to visualize anything D3 can adapt to all projects needs. As D3 allows for the full control of appearance and behavior of the diagrams, it can be adapted to any existing style guides. When only simple diagrams are needed, which do not need to react to data changes in real-time and instead only fetch data once as the site is loaded, the initial learning curve of D3 is also not as steep. If the visualizations should be more complex, update with live data changes and follow specific specifications, D3 is still a very well suited tool. In both cases it is worth using D3. Simple diagrams allow for an easy beginning to get familiar with the tool. Once familiar, D3 is capable to work with more complex visualizations.

Furthermore, as the current situation is the Ukraine is not always so clear, it is hard to acquire accurate refugee data. This is also mentioned on the UNHCR situation page, were it is stated that "Statistics are compiled mainly from data provided by authorities. While every effort has been made to ensure that all statistical information is verified, figures represent an estimate. Triangulation of information and sources is performed on a continuous basis. Therefore, amendments to figures may occur, including retroactively."\cite{unhcr}. Allowing the user to manually manipulate data in the showcase finally removes all credibility of the shown data. Yet this does not influence the conclusion of this thesis, as this thesis is focussed on D3. Not on accurately representing the current refugee situation.