\chapter{Discussion}

As there are three main questions this thesis aims to answer, each question will be discussed separately. Beginning with the question of what D3 can do. Looking at the created showcase, D3 can obviously be used to create many different types of diagrams. Looking at the examples found online, D3 has their own showcase of projects, makes this even more apparent. From simple bar and pie charts, over visualizing hierarchical data using tree maps or Sankey diagrams, like the ones created for this thesis, all the way to map based diagrams using various projections\cite{davies}, physics enabled bubble graphs\cite{carter_2012} and pseudo 3D animations\cite{davies_sphere}. Of course D3 can also be used to created animations which do not necessarily serve data visualization purposes, like the tadpoles example\cite{bostock_2020}. Due to D3's low level approach, fast speed, and the general update pattern, D3 can be used to create all visualizations one can imagine and have them react to data changes in real time. As D3 is built around simple DOM manipulation, it could also be used for other aspects. One could draw charts and populate them with data. One can make small animations to add visually appealing aspects to a website. Or maybe one can adapt the scales for their own needs of converting data. But what does D3 excel at and where does it struggle?

When working in a web environment, it is always easy to start using D3 as well. Its independence from any framework and implementation only in JavaScript, makes it possible to include D3 in any web-based project. On the other hand, if one is not already working on a web-based project but still wants to use D3, one has to deal with all the additional overhead of working with a web technology stack. Whilst importing the D3 library is really easy, the initial learning curve is everything but easy. Without first internalizing the core concepts of D3 and SVG, it is impossible to make any kind of visualization. While this is true for any framework or technology, the concept of the general update pattern seems especially abstract. Yet it is crucial to understand it when a diagram is supposed to react to data changes.
While learning the basics of D3 is quite a big hurdle, it can be broken down by first creating only static diagrams. This can be achieved without a deeper understanding of the general update pattern. While D3s low-level approach is cumbersome tp comprehend initially, it actually allows D3 to be very flexible.
In addition, while there are a lot of examples and tutorials, they are often too complex to understand as a beginner, or use varying versions of D3 or JavaScript styles. Besides some functionalities of D3 being obsolete in newer versions, the different styles of JavaScript can be additionally confusing when the developer is not familiar with the evolution of JavaScript.
Once one understands how D3 works, it can be quite fast to create basic diagrams. Once one properly understands the general update pattern and selections, it is also not too difficult to react to data changes. Yet trying to animate diagrams can become tricky. It depends on the elements which are used and which attributes need to be animated. As all elements have to be manually specified, one has full control over the appearance and behavior of the diagrams. But having to specify all aspects so precisely is also time-consuming.
Another concern might be the performance of D3 when dealing with many marks in a diagram. Therefore the limits of smooth animation of the bar-chart and the tadpoles example\cite{bostock_2020} were briefly tested. Of course these results vary from device to device. In this case, it was possible to smoothly animate up to approximately 1200-1300 tadpoles. The bar-chart is harder to evaluate. The singular bars approach, and partially pass, the limits of a singular pixel in thickness, and therefore their visibility on the screen, with this number of entries in the diagram. While this thesis does not provide a proper performance test and comparison with other visualization libraries performances, it can be safely assumed that performance will not be an issue for most diagrams.

So when should one use D3? This depends on the task at hand. If the goal is to create some diagrams as a one time job, D3 is unnecessarily complex. Tools like Excel can easily excel here. Even when working with a web project, D3 is probably too complex. Most developers will find all necessarily functionality using libraries like Chart.js\cite{chartjs} or the Plotly\cite{plotly} JavaScript library, which is actually built on top of D3 to offer a more high level approach to data visualization. Both these open-source libraries allow creating some of the most common diagrams without having to learn all the quirks of D3.
Yet if someone is to create a custom visualization or wants more fine control over all aspects of a diagram, D3 can handle it. Due to the low level approach and easy implementation with other frameworks, D3 can used to visualize virtually anything. As D3 allows for the full control of appearance and behavior of the diagrams, it can also be adapted to any existing style guides. On the other hand, this also allows the freedom to create diagram with a bad data-ink ratio or a high lie factor. This risk can be mitigated by using a more high level library.
Due to the high initial learning curve, getting into D3 can only be recommended when the creating custom visualizations, which high level libraries can not provide.
