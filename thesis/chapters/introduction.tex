\chapter{Introduction}
%\textcolor{red}{
%Describe the background of the thesis, why it is important, what do we want to achieve.
%Why is this thesis (Motivation)? What do we want to do? What is the status quo? What benefits will result from this thesis?
%Something about the importance of infographics and comprehendible data.}


The postmodern world produces huge amounts of data every second. Analyzing this data can lead to better-informed decision-making in every sector. Yet the wast amounts of gathered data is often hard to comprehend with the human mind. Data visualization is about finding ways to represent this data in visually appealing and easily comprehendible ways\cite{sadiku2016data}. Doing this quickly, ideally instant, and being always up to date can be crucial. While it is possible to create data visualizations manually, it is common to use computer tools to help in their creation. There are many tools available to help with the creation of diagrams for data visualization. Some of these tools have a graphical-user-interface, like Excel\cite{excel}, others are code based, like R\cite{r} or the Matplotlib\cite{matplotlib} library for Python. As the requirements for a data visualization project can vary, it is often not easy to decide which tool best suits ones needs. Therefore this thesis will be a deep dive into the broad possibilities of one of these code based tools, the 'd3.js'(D3) library for JavaScript. Whilst there is a lot of information and examples on how to use D3, the available information makes use of a variety of code styles and different versions of D3. This makes it hard to properly evaluate the possibilities of D3 as a data visualization tool. Yet knowing when to use which tool can be greatly beneficial for all parties involved.

To evaluate D3 and its possibilities there are three main questions that will be investigated in this thesis. What is the potential of D3 in data visualization? What are the advantages and disadvantages of using D3? When is it reasonable to use D3? To be able to evaluate these questions and get a better understanding of D3, a showcase of several different diagrams is created and evaluated throughout this thesis. This showcase is created using and visualizing refugee data of the currently ongoing Ukraine conflict. A live version of the showcase can be found at "\texttt{https://styxoo.github.io/}".

All the necessary theoretical background necessary to understand and follow the thesis is described in chapter 2. Afterwards the implementation of the showcase and the diagrams which are contained, is described in chapter 3. Finally in chapter 4 the results are discussed, before drawing a personal conclusion in chapter 5.
