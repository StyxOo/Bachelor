\chapter{Introduction}
Describe the background of the thesis, why it is important, what do we want to achieve.
Why is this thesis (Motivation)? What do we want to do? What is the status quo? What benefits will result from this thesis?
Something about the importance of infographics and comprehendible data. 


The postmodern world produces huge amounts of data every second. Analyzing this data can lead to better-informed decision-making in every sector. Yet the wast amounts of created data is often hard to comprehend with the human mind. Data visualization is about finding ways to represent this data in visually appealing and easily comprehendible representations. Doing this quickly and always up to date can be crucial. While it is possible to create data visualizations manually it is more common nowadays to use computer tools to help in their creation. There are many tools available to help with the creation of infographics. Some of the data visualization tools have a graphical-user-interface, others are code based. It is often not easy to decide which tool best suits ones needs. Therefore this thesis will be a deep dive into the broad possibilities of one of these tools, the 'd3.js'(D3) library for JavaScript. Whilst there is a lot of information on how to use D3, there is little information to be found on when it is reasonable to use D3. Yet knowing when to use which tool is greatly beneficial for all parties involved. To achieve this there are two main questions that will be answered in this thesis. What is the potential of D3 in data visualization? What are the advantages and disadvantages of using D3? To be able to evaluate these questions a showcase of a several different diagrams is created.
