\chapter{Conclusion}
%\textcolor{red}{
%How well did it work? Was it worth the effort? What could be improved?}

Learning and understanding the core concepts of D3 took a surprisingly long time. While there are many examples, the inconsistency in D3 and JavaScript versions were quite confusing. Having never worked with either so intensely, it took a fairly long time to get used to it. A few parts have been especially cumbersome. While animating updates with transitions is usually easy, creating a smooth animation for the donut chart took longer than expected. Having to work with the custom attribute tweens and storing information on the DOM element itself, made this even more confusing. I was also unable to animate the area-graph in a way where the date-line follows the line along the top of the area. This is something which can most certainly be done. But not by me in the time span of creating this thesis. Another weird issues for which I still do not have an idea of how to fix it, is the legend of the circle-graph. It behaves in unexpected ways when the data is changed to include a maximum value greatly higher than the original values.
Creating the right selections and sub-selections when implementing the general-update pattern also is not always easy. While drawing a diagram initially is usually an easily achievable feat, making sure that update behavior reuses existing elements is sometimes tricky. The widespread use of D3 at least helped finding information on common issues, which was very helpful for bug fixing.

While the implementation doesn't differ for using discrete and continuous data, it would still have been nice to show this in an actual example. Another interesting aspect for which I did not have the time to implement, is working maps and projections. A map could also have easily shown the refugee streams.

While learning the basics of D3 is quite a big hurdle, it can be broken down by first creating only static diagrams. This can be achieved without a deeper understanding of the general update pattern. When understanding the general update pattern afterwards, one can suddenly see the enormous potential of D3. While the low-level approach might at first seem cumbersome, it actually allows D3 to be a lot more flexible. 

Due to the currentness of the data, the UNHCR was updating their situation page as well. Therefore the terminology which was used changed from initially mentioning refugees, to later solely mentioning border-crossings. As the diagrams and showcase were implemented before writing the text for this thesis, the renaming was glossed over, as all the created work would have to be redone. This is also why the code files always mention refugees. Furthermore the data presented was changed as well. This lead to the inconsistent end dates for both data-sets, as the UNHCR stopped listing the refugees aka border-crossings per day. It is very interesting to see the quite steady increase of total refugees. I would have expected a huge influx in the beginning and a flattening curve afterwards, as most people wanting to leave have left. Especially as the beginning of the conflict saw fighting more widespread throughout the country. While the current situation is focussed more on the east.
