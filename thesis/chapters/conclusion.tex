\chapter{Conclusion}\label{sec:conclusion}
%\textcolor{red}{
%How well did it work? Was it worth the effort? What could be improved?}

While D3 is an immensely powerful tool, learning and understanding the core concepts of D3 took a surprisingly long time. While there are many examples, the inconsistencies in D3 versions as well as JavaScript versions were quite confusing. Having never worked with D3 and only having very limited experience with JavaScript, it took a fairly long time to get used to both.
Even now, if I was tasked to create some simple diagrams, I would probably use a more high-level library. Unless the required diagrams require the full potential of D3.
A few parts have been especially cumbersome. While animating updates with transitions is usually easy, creating a smooth animation for the donut chart took longer than expected. Having to work with the custom attribute tweens and storing information on the DOM element itself, made this even more confusing. Yet this was in part due to my lacking proficiency in working with JavaScript and the differences between function declarations using arrow functions and the function keyword. I was also unable to animate the area graph in a way where the date line follows the line along the top of the area. Furthermore it would be nice if the date line would be draggable using the mouse. This is something which can most certainly be done. But not by me in the time span of creating this thesis. Creating the right selections and sub-selections when implementing the general update pattern also is not always easy. While drawing a diagram initially is usually an easily achievable feat, making sure that update behavior reuses existing elements is sometimes tricky. The widespread use of D3 at least helped finding information on common issues, which was very helpful for bug fixing.

While the implementation doesn't differ for using discrete and continuous data, it would still have been nice to show this in an actual example. Another interesting aspect which is not explored by this thesis, is working with maps and projections. A map could also have easily shown the refugee streams.

Due to the currentness of the data, the UNHCR was updating their situation page as well. On of the effects of this was that the terminology they used changed from initially mentioning refugees, to later solely mentioning border crossings. This can actually still be seen when looking at the raw data JSON file, which contains the cumulative number of refugees per day. It contains a description clearly mentioning the data being about refugees. As the diagrams and showcase were implemented before writing the text for this thesis, the renaming was glossed over, as all the created work would have to be redone. This is also why the code files always mention refugees. 

Finally, while it was interesting to work with D3 and get to know its immense potential, it is devastating to see that, yet again, millions of people have been displaced by unnecessary violence and aggression.
